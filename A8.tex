\documentclass{beamer}

\usepackage{xspace}

 \usepackage{calc}                                             %%
    \usepackage{multirow}                                         %%
    \usepackage{hhline}                                           %%
    \usepackage{ifthen}   
    
\newcounter{saveenumi}
\newcommand{\seti}{\setcounter{saveenumi}{\value{enumi}}}
\newcommand{\conti}{\setcounter{enumi}{\value{saveenumi}}}                                      

\setbeamertemplate{caption}[numbered]{}
    
\providecommand{\pr}[1]{\ensuremath{\Pr\left\{#1\right\}}}
\providecommand{\brak}[1]{\ensuremath{\left(#1\right)}}
\providecommand{\cbrak}[1]{\ensuremath{\left\{#1\right\}}}
\providecommand{\sbrak}[1]{\ensuremath{{}\left[#1\right]}}


% Theme choice:
\usetheme{CambridgeUS}

% Title page details: 
\title{ASSIGNMENT 8 : PAPOULLIS CHAPTER : 3 EXAMPLE - 3-3} 
\author{AI21BTECH11016}
\date{\today}
\logo{\large \LaTeX{}}


\begin{document}

% Title page frame
\begin{frame}
    \titlepage 
\end{frame}

% Remove logo from the next slides
\logo{}

% Outline frame
\begin{frame}{Outline}
    \tableofcontents
\end{frame}

% Outline frame
\section{Question}
\begin{frame}{Question}
\begin{block}{}
A box B1 contains 10 white and 5 red balls and a box B2 contains 20 white and
20 red balls. A ball is drawn from each box. What is the probability that the ball from
B1 will be white and the ball from B2 red?
\end{block}
\end{frame}

\section{Declaration of Random Variables}
\begin{frame}{Solution}
Let X = \cbrak{0,1} be a random variable representing the box from which ball is drawn.\\
Let Y = \cbrak{0,1} be a random variable representing the colour of the ball.\\

\begin{block}{}
     \begin{table}[ht!]
    \centering
    \input{tables/table8.tex}
    \caption{Random Variables}
    \label{Tab:1}
\end{table}   
    \end{block}
    
\end{frame}

\section{Calculation of Probabilities}
\begin{frame}
\begin{enumerate}
\item{
Probability that the ball from B1 will be white :
\begin{align}
\pr{\brak{X = 0} \brak{Y = 0}} & = \frac{10}{15}\\ & = \frac{2}{3}
\end{align}}

\item{
Probability that the ball from B2 will be red :
\begin{align}
\pr{\brak{X = 1} \brak{Y = 1}} & = \frac{20}{40}\\ & = \frac{1}{2}
\end{align}}
\seti
\end{enumerate}
\end{frame}


\begin{frame}
\begin{enumerate}
\conti
\item{
Probability that the ball from B1 will be white and the ball from B2 red is
\begin{align}
\pr{\sbrak{\brak{X = 0} \brak{Y = 0}} \sbrak{\brak{X = 1} \brak{Y = 1}}} 
\end{align}

Clearly, the events of drawing a ball from each box are independent i.e.,
\begin{align}
\Rightarrow \pr{\brak{X = 0} \brak{Y = 0}} \pr{\brak{X = 1} \brak{Y = 1}} & = \frac{2}{3} \times \frac{1}{2}\\ & = \frac{1}{3}
\end{align}}
\end{enumerate}
\end{frame}

\section{Answer}
\begin{frame}
\begin{block}{}
$\therefore$ Probability that the ball drawn from B1 is white and B2 is red is\\
\begin{align}
\pr{\sbrak{\brak{X = 0} \brak{Y = 0}} \sbrak{\brak{X = 1} \brak{Y = 1}}} & = \frac{1}{3}.
\end{align}
\end{block}
\end{frame}

\end{document}